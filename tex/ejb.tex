% !TEX root = ejb-web.tex
%!TEX encoding = UTF-8 Unicode
\ifx\setbeamertemplate\undefined
\documentclass[handout]{beamer}
\fi
\graphicspath{{../images/}}
\usepackage{amsmath}
\usepackage{xifthen}

\usepackage{mathtools}
\mathtoolsset{showonlyrefs}
\usepackage{mathrsfs}

\usepackage[multidot]{grffile}

\usepackage{xspace}
\usepackage{upgreek}
\newcommand\xsb{$\upchi$SB\xspace}
\newlength{\imagecolumnwidth}

\usepackage[absolute,overlay]{textpos}

\usepackage{slashed}
\newcommand\dslash{\slashed{\partial}}

\usepackage{bbold}
\newcommand\one{\mathbb{1}}

\usepackage{multirow}

\newcommand\bs\boldsymbol

\newcommand \Sx[1] {S_{\textnormal{#1}}}
\newcommand \Sg[2] {\mathrm{#1}(#2)}
\newcommand \SU[1] {\Sg{SU}{#1}}
\newcommand \SO[1] {\Sg{SO}{#1}}
\newcommand \Sp[1] {\Sg{Sp}{#1}}
\newcommand \Uone {\Sg{U}{1}}

\newcommand\pb{\overline{\psi}}
\newcommand \bigL {\mathscr{L}}
\newcommand \lag[1] {\bigL_{\mathrm{#1}}}

\newcommand\sptn{Sp($2N$)\xspace}
\newcommand\dee{\mathrm{d}}
\newcommand\eee{\mathcal{E}}
\newcommand\www{\mathcal{W}}
\DeclareMathOperator\tr{tr}
\DeclareMathOperator\real{Re}
\newcommand\Nf{N_{\mathrm{f}}}

%\usepackage{beamerthemesplit} %Activate for custom appearance
\setbeamertemplate{navigation symbols}{}%remove navigation symbols

\usefonttheme[onlymath]{serif}
\usepackage{fontspec,xltxtra,xunicode}
\defaultfontfeatures{Mapping=tex-text}
\setsansfont[Scale=MatchLowercase,Mapping=tex-text]{Futura}
\setbeamerfont{structure}{family=\fontspec{Cosmos BQ}}
\definecolor{swanseablue}{RGB}{36,47,96}
\setbeamercolor{structure}{fg=swanseablue}
 \setbeamertemplate{itemize item}{•}
 \setbeamertemplate{itemize subitem}{--}
 \setbeamercolor{itemize item}{fg=black}
 \setbeamercolor{itemize subitem}{fg=black}


\newfontfamily{\J}[Scale=0.85]{Osaka}
\newcommand\module[1][]{\texttt{module\ifthenelse{\equal{#1}{}}{}{ #1}}\xspace}

\newcommand\disappear[1]{}
\title{What even \emph{is} the \module command?}
\author{{\large Ed Bennett}
	\\{\small@QuantumofEd}
	\\\vspace{16pt}
	\hfill
	\parbox{0.22\textwidth}{\centering\includegraphics[height=36pt]{logos/swansea}\\\small @SwanseaUni}
	\parbox{0.44\textwidth}{\centering\includegraphics[height=24pt]{logos/scw}\\\small @SuperCompWales}
	\parbox{0.22\textwidth}{\centering\includegraphics[height=36pt]{logos/sa2c}\\\small @sa2c\_swansea} }
\date{SA2C Tech Chat, 2018-07-27}

\newcommand\Wider[2][3em]{%
\makebox[\linewidth][c]{%
  \begin{minipage}{\dimexpr\paperwidth-#1\relax}
  \raggedright#2
  \end{minipage}%
  }%
}



\newcommand\FrameText[1]{%
  \begin{textblock*}{0.9\paperwidth}(3pt, 0.99\textheight)
    \raggedright #1\hspace{.5em}
  \end{textblock*}}

\newcommand \sechead[1] {\frame{\vfill\Huge \centering\usebeamerfont{structure}\color{swanseablue}#1\vfill\null}}

\usebackgroundtemplate%
{%
	\includegraphics[width=\paperwidth,height=\paperheight]{background/bg}%
}
\setbeamertemplate{footline}{
	\vspace{1.0cm}
}


  
\begin{document}

\frame{\titlepage}

\begin{frame}{\module basics}

	\begin{itemize}[<+->]
		\item HPC clusters provide a lot of software
		\item Not all of it can be in the environment at once
		\item Install each package in its own folder
		\item Add to the \texttt{\$PATH} on request
	\end{itemize}
\end{frame}

\begin{frame}{Using \module}

	\begin{itemize}[<+->]
		\item \module[avail]
		\item \module[load]
		\item \module[list]
		\item \module[unload]
		\item \module[purge]
	\end{itemize}
\end{frame}

\begin{frame}[fragile]{Promises}
	``Modules can be loaded and unloaded dynamically and atomically, in an clean fashion.''\pause
	\begin{verbatim}
[s.e.j.bennett@cl2 ~]$ echo $F90\end{verbatim}\pause\begin{verbatim}

[s.e.j.bennett@cl2 ~]$ export F90=gfortran\end{verbatim}\pause\begin{verbatim}
[s.e.j.bennett@cl2 ~]$ module load compiler/intel\end{verbatim}\pause\begin{verbatim}
[s.e.j.bennett@cl2 ~]$ echo $F90\end{verbatim}\pause\begin{verbatim}
ifort
[s.e.j.bennett@cl2 ~]$ module unload compiler/intel\end{verbatim}\pause\begin{verbatim}
[s.e.j.bennett@cl2 ~]$ echo $F90\end{verbatim}\pause\begin{verbatim}

[s.e.j.bennett@cl2 ~]$
\end{verbatim}

\end{frame}

\begin{frame}{Wait a minute!}

	\begin{itemize}[<+->]
		\item Executable scripts run in a subshell
		\item They can't change the environment
		\item To change the environment needs \texttt{source} or \texttt{.}, to run in the current shell
		\item What even \emph{is} \module?
	\end{itemize}
\end{frame}

\begin{frame}[fragile]{Is it an alias?}

	\pause
	\begin{verbatim}
[s.e.j.bennett@cl2 ~]$ which module\end{verbatim}\pause\begin{verbatim}
/usr/bin/which: no module in (/usr/lib64/qt-3.3/bin:
/usr/local/bin:/usr/bin:/usr/local/sbin:/usr/sbin:
/home/s.e.j.bennett/bin)
\end{verbatim}

	\pause
	\begin{itemize}[<+->]
		\item Nope
		\item Where \emph{is} it!?
	\end{itemize}
\end{frame}

\begin{frame}[fragile]{Aside}
	
	\begin{verbatim}		
[s.e.j.bennett@cl2 ~]$ module avail | grep intel\end{verbatim}\pause\begin{verbatim}
-------- /apps/modules/legacy --------
hpcw  raven
...
\end{verbatim}

	\begin{itemize}[<+->]
		\item \module outputs \emph{everything} to \texttt{stderr}
		\item Why?
			
	\end{itemize}

\end{frame}

\begin{frame}[fragile]{Progress}
	
	\begin{verbatim}
[s.e.j.bennett@cl2 ~]$ cat /etc/profile.d/modules.sh\end{verbatim}\pause\begin{verbatim}
shell=`/bin/basename \`/bin/ps -p $$ -ocomm=\``
if [ -f /usr/share/Modules/init/$shell ]
then
. /usr/share/Modules/init/$shell
else
. /usr/share/Modules/init/sh
fi\end{verbatim}
\end{frame}


\begin{frame}[fragile]{Eureka}
	
	\begin{verbatim}
[s.e.j.bennett@cl2 ~]$ cat /usr/share/Modules/init/sh\end{verbatim}\pause\begin{verbatim}
module() { eval `/usr/bin/modulecmd sh $*`; }

MODULESHOME=/usr/share/Modules
export MODULESHOME

...
\end{verbatim}

	\begin{itemize}[<+->]
		\item So \module is a \emph{function}
		\item Now, how does it work?
	\end{itemize}
\end{frame}


\begin{frame}{\texttt{modulecmd}}
	\begin{itemize}[<+->]
		\item 7392 lines of Tcl \uncover<4->{(https://git.io/fNEgB)}
		\item ...
		\item Let's look at a modulefile instead
	\end{itemize}
\end{frame}

\begin{frame}[fragile]{Finding some module files}
	
	\begin{verbatim}[s.e.j.bennett@cl2 ~]$ cat /etc/profile.d/02-default-modules.sh\end{verbatim}\pause\begin{verbatim}
#!/bin/sh

# Ansible managed

module use /apps/modules/physics
module use /apps/modules/medical
module use /apps/modules/materials
module use /apps/modules/genomics
module use /apps/modules/financial
...\end{verbatim}

\end{frame}


\begin{frame}[fragile]{Module file directory structure}
	
	\begin{verbatim}
[s.e.j.bennett@cl2 ~]$ find /apps/modules/\end{verbatim}\pause\begin{verbatim}
/apps/modules/
/apps/modules/legacy
/apps/modules/legacy/raven
/apps/modules/legacy/hpcw
/apps/modules/medical
/apps/modules/materials
/apps/modules/materials/fluidity
/apps/modules/materials/OpenFOAM
/apps/modules/materials/OpenFOAM/5.x-20180613
/apps/modules/materials/QuantumEspresso
/apps/modules/materials/QuantumEspresso/6.1
...
\end{verbatim}

\end{frame}


\begin{frame}[fragile]{OK OK, show me a modulefile already!}
	
	\begin{verbatim}
[s.e.j.bennett@cl2 ~]$ more /app/modules/materials/cp2k/2.4-intel-mpi\end{verbatim}\pause\begin{verbatim}
#%Module1.0#####################################################################
source /app/modules/system/logger.tcl
##
## CP2K modulefile
##
set appname "cp2k"
set version "2.4"
set para_ver "mpi"
set type "popt"
--More--
\end{verbatim}

	\begin{itemize}[<+->]
		\item Modulefiles are Tcl programs
		\item \verb|modulecmd| executes these, and does stuff with the result
	\end{itemize}

\end{frame}

\begin{frame}[fragile]{More of this cp2k modulefile}
	
	\begin{verbatim}
proc ModulesHelp { } {
  puts stderr "\tLoads PATH and LD_LIBRARY_PATH settings for cp2k.\n"
  puts stderr "\tAn example jobscript and input file can be found in the cp2k_example.tar.gz file."
  puts stderr "\tThis, along with a README file, is in the example directory in the cp2k application directory."
  puts stderr "\tThe cp2k application directory can be found by loading the cp2k module and then"
  puts stderr "\tlooking at the output of `which cp2k` and replacing 'bin/cp2k' with 'example'.\n"
  puts stderr "\tMore information on this package can be found at http://www.cp2k.org\n"
}
--More--
\end{verbatim}

	\begin{itemize}[<+->]
		\item Module help gets output to \texttt{stderr}
		\item Why?
		\item Let's keep going...
	\end{itemize}
\end{frame}


\begin{frame}[fragile]{Oh no...}
	
	\begin{verbatim}
  # Set ulimit unlimited
   if { [module-info shelltype sh] } {
     puts "ulimit -s unlimited;"
   } elseif { [module-info shelltype csh] } {
     puts "limit stacksize unlimited;"
   }
--More--
\end{verbatim}

	\begin{itemize}[<+->]
		\item Why are we \verb|puts|ing shell commands to \verb|stdout|?
		\item Wait
		\item Surely not
	\end{itemize}
\end{frame}

\begin{frame}[fragile]{}{}
	{\LARGE \begin{verbatim}
module() { 
    eval `/usr/bin/modulecmd sh $*`; 
}
\end{verbatim}

}
\end{frame}

\begin{frame}{\texttt{eval} considered harmful (to sanity)}
\begin{itemize}[<+->]
	\item All actual work is done inside \texttt{modulecmd}
	\item The output of which is passed straight to \texttt{eval}
	\item Commands are run by outputting them to \texttt{stdout}
	\item So messages can only be displayed via \texttt{stderr}
	\item We can cross-check this in the \texttt{modulecmd} source...
\end{itemize}
\end{frame}

\begin{frame}[fragile]{\texttt{modulecmd} revisited}
\begin{verbatim}
switch -- $::g_shellType {
...
{sh} {
puts stdout "$var=[charEscaped $::env($var)];\
export $var;"
}
...
{tcl} {
set val $::env($var)
puts stdout "set ::env($var) {$val};"
}
{cmd} {
set val $::env($var)
puts stdout "set $var=$val"
}	
..
\end{verbatim}

\end{frame}

\begin{frame}[fragile]{Private modules}
	\begin{itemize}[<+->]
		\item Recall \verb|module use /apps/modules/physics|
		\item Last argument is just a path
		\item Paths are scanned for modulefile-like things
		\item So we can add a path in \verb|$HOME|
		\item This is built in via the \verb|use.own| module, which looks in \verb|$HOME/privatemodules|
	\end{itemize}
\end{frame}


\begin{frame}[fragile]{Writing our own module}
	\begin{verbatim}
#%Module -*- tcl -*-
## This is a module to access something
proc ModulesHelp { } {
  puts stderr "This module sets up access to something" 
}
module-whatis "sets up access to something"
prereq something/else
conflict that/other/module
module load gcc
setenv       SOMEVERION       0.95
append-path  PATH             /home/[user]/[somedir]/bin
append-path  MANPATH          /home/[user]/[somedir]/man
append-path  LD_LIBRARY_PATH  /home/[user]/[somedir]/lib
\end{verbatim}
\end{frame}


\begin{frame}{Thanks for listening!}
	
	\Wider[0em]{\includegraphics[width=\textwidth]{figs/swansea-panorama}}
\end{frame}

\end{document}
